% =================================================================================================
% File:			nome_del_capitolo.tex
% Description:	Defiinisce la sezione relativa a ...
% Created:		2014-12-05
% Author:		Santacatterina Luca
% Email:		s88.luca@gmail.com
% =================================================================================================
% Modification History:
% Version		Modifier Date		Change											Author
% 0.0.1 		2014-12-05 			iniziata stesura documento di prova				Luca S.
% =================================================================================================
%

% CONTENUTO DEL CAPITOLO

\section{Capitolato C5}
\subsection{Informazioni sul capitolato}

\begin{itemize}
    \item Nome: P2PCS, piattaforma di peer-to-peer car sharing
	\item Proponente: GaiaGo
	\item Committente: Prof. Tullio Vardanega e Prof. Riccardo Cardin
\end{itemize}

\subsection{Descrizione}
Lo scopo del capitolato proposto è quello di creare un'applicazione in ambiente Android che permetta di condividere la propria macchina con altre persone.

\subsection{Dominio Applicativo}
Le problematiche affrontate da questo capitolato risultano sicuramente attuali. Il car sharing viene ampiamente utilizzato al giorno d'oggi , ma il servizio viene perlopiù fornito da aziende private. Questo progetto consentirebbe principalmente alle persone che possiedono autovetture inutilizzate(ma anche non)di avere degli introiti, a fronte delle spese per il mantentimento dell'autovettura stessa.
\subsection{Dominio Tecnologico}
\begin{itemize}
    \item Android come piattaforma per cui sviluppare
    \item Node.js, Goodle Cloud
    \item Architettura di rete peer-to-peer
\end{itemize}

\subsection{Aspetti positivi}
\begin{itemize}
    \item Nessuno
\end{itemize}
\subsection{Fattori di rischio}
\begin{itemize}
    \item Lo sviluppo in Ambiente Android non ricopre l'intero mercato mobile
    \item fa schifo aa
\end{itemize}
\subsection{Conclusioni}
Questro progetto fa cagare


